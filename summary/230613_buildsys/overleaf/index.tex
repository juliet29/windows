% Options for packages loaded elsewhere
\PassOptionsToPackage{unicode}{hyperref}
\PassOptionsToPackage{hyphens}{url}
\PassOptionsToPackage{dvipsnames,svgnames,x11names}{xcolor}
%
\documentclass[
  letterpaper,
  DIV=11,
  numbers=noendperiod]{scrartcl}

\usepackage{amsmath,amssymb}
\usepackage{lmodern}
\usepackage{iftex}
\ifPDFTeX
  \usepackage[T1]{fontenc}
  \usepackage[utf8]{inputenc}
  \usepackage{textcomp} % provide euro and other symbols
\else % if luatex or xetex
  \usepackage{unicode-math}
  \defaultfontfeatures{Scale=MatchLowercase}
  \defaultfontfeatures[\rmfamily]{Ligatures=TeX,Scale=1}
\fi
% Use upquote if available, for straight quotes in verbatim environments
\IfFileExists{upquote.sty}{\usepackage{upquote}}{}
\IfFileExists{microtype.sty}{% use microtype if available
  \usepackage[]{microtype}
  \UseMicrotypeSet[protrusion]{basicmath} % disable protrusion for tt fonts
}{}
\makeatletter
\@ifundefined{KOMAClassName}{% if non-KOMA class
  \IfFileExists{parskip.sty}{%
    \usepackage{parskip}
  }{% else
    \setlength{\parindent}{0pt}
    \setlength{\parskip}{6pt plus 2pt minus 1pt}}
}{% if KOMA class
  \KOMAoptions{parskip=half}}
\makeatother
\usepackage{xcolor}
\setlength{\emergencystretch}{3em} % prevent overfull lines
\setcounter{secnumdepth}{-\maxdimen} % remove section numbering
% Make \paragraph and \subparagraph free-standing
\ifx\paragraph\undefined\else
  \let\oldparagraph\paragraph
  \renewcommand{\paragraph}[1]{\oldparagraph{#1}\mbox{}}
\fi
\ifx\subparagraph\undefined\else
  \let\oldsubparagraph\subparagraph
  \renewcommand{\subparagraph}[1]{\oldsubparagraph{#1}\mbox{}}
\fi


\providecommand{\tightlist}{%
  \setlength{\itemsep}{0pt}\setlength{\parskip}{0pt}}\usepackage{longtable,booktabs,array}
\usepackage{calc} % for calculating minipage widths
% Correct order of tables after \paragraph or \subparagraph
\usepackage{etoolbox}
\makeatletter
\patchcmd\longtable{\par}{\if@noskipsec\mbox{}\fi\par}{}{}
\makeatother
% Allow footnotes in longtable head/foot
\IfFileExists{footnotehyper.sty}{\usepackage{footnotehyper}}{\usepackage{footnote}}
\makesavenoteenv{longtable}
\usepackage{graphicx}
\makeatletter
\def\maxwidth{\ifdim\Gin@nat@width>\linewidth\linewidth\else\Gin@nat@width\fi}
\def\maxheight{\ifdim\Gin@nat@height>\textheight\textheight\else\Gin@nat@height\fi}
\makeatother
% Scale images if necessary, so that they will not overflow the page
% margins by default, and it is still possible to overwrite the defaults
% using explicit options in \includegraphics[width, height, ...]{}
\setkeys{Gin}{width=\maxwidth,height=\maxheight,keepaspectratio}
% Set default figure placement to htbp
\makeatletter
\def\fps@figure{htbp}
\makeatother

\KOMAoption{captions}{tableheading}
\makeatletter
\makeatother
\makeatletter
\makeatother
\makeatletter
\@ifpackageloaded{caption}{}{\usepackage{caption}}
\AtBeginDocument{%
\ifdefined\contentsname
  \renewcommand*\contentsname{Table of contents}
\else
  \newcommand\contentsname{Table of contents}
\fi
\ifdefined\listfigurename
  \renewcommand*\listfigurename{List of Figures}
\else
  \newcommand\listfigurename{List of Figures}
\fi
\ifdefined\listtablename
  \renewcommand*\listtablename{List of Tables}
\else
  \newcommand\listtablename{List of Tables}
\fi
\ifdefined\figurename
  \renewcommand*\figurename{Figure}
\else
  \newcommand\figurename{Figure}
\fi
\ifdefined\tablename
  \renewcommand*\tablename{Table}
\else
  \newcommand\tablename{Table}
\fi
}
\@ifpackageloaded{float}{}{\usepackage{float}}
\floatstyle{ruled}
\@ifundefined{c@chapter}{\newfloat{codelisting}{h}{lop}}{\newfloat{codelisting}{h}{lop}[chapter]}
\floatname{codelisting}{Listing}
\newcommand*\listoflistings{\listof{codelisting}{List of Listings}}
\makeatother
\makeatletter
\@ifpackageloaded{caption}{}{\usepackage{caption}}
\@ifpackageloaded{subcaption}{}{\usepackage{subcaption}}
\makeatother
\makeatletter
\@ifpackageloaded{tcolorbox}{}{\usepackage[many]{tcolorbox}}
\makeatother
\makeatletter
\@ifundefined{shadecolor}{\definecolor{shadecolor}{rgb}{.97, .97, .97}}
\makeatother
\makeatletter
\makeatother
\ifLuaTeX
  \usepackage{selnolig}  % disable illegal ligatures
\fi
\IfFileExists{bookmark.sty}{\usepackage{bookmark}}{\usepackage{hyperref}}
\IfFileExists{xurl.sty}{\usepackage{xurl}}{} % add URL line breaks if available
\urlstyle{same} % disable monospaced font for URLs
\hypersetup{
  colorlinks=true,
  linkcolor={blue},
  filecolor={Maroon},
  citecolor={Blue},
  urlcolor={Blue},
  pdfcreator={LaTeX via pandoc}}

\author{}
\date{}

\begin{document}
\ifdefined\Shaded\renewenvironment{Shaded}{\begin{tcolorbox}[boxrule=0pt, frame hidden, borderline west={3pt}{0pt}{shadecolor}, enhanced, interior hidden, breakable, sharp corners]}{\end{tcolorbox}}\fi

\hypertarget{introduction}{%
\section{Introduction}\label{introduction}}

\href{file:///Users/julietnwagwuume-ezeoke/_UILCode/windows/summary/230530_pres/UILpres.html\#/mechanical-cooling}{Windows
presentation}

As the planet warms, there is an increased need for urgency in both
mitigating and adapting the effects of climate change. This urgency is
motinvated by the significant uptick in the frequency, duration and
cumulative intensity of heat waves, which have climbed from \#TODO X to
X @perkins-kirkpatrickIncreasingTrendsRegional2020. The significant
health risks posed by heat waves, including heat exhaustion, stroke, and
increased risk for those with pre-existing conditions have resulted in
over 160,00 excess deaths between 1998 and 2017 @whoHeatwaves2023.

To cope with increased heat, attention is often directed at air
conditioners and other forms of mechanical cooling. These have the
advantage of being relatively easy to design for -- one simply has to
estimate the cooling loads and traslate them into power requirements
when sizing a system. The codes surrounding mechanical cooling are well
established, and oweners and engineers alike take on little risk when
integreating a mechanical cooling system into a building. Developments
in mechanical cooling have made portable air conditioners readily
available for people who occupy buildings (\#TODO housing? need to
discuss residential focus) that do not have pre-installed systems.

However, while mechanical cooling is often percieved as reliable, it is
susceptible to failure when it is most needed -- during heatwaves. In a
community that relies on mechanically cooled systems, electricity demand
will increase markedly during heat waves. However, mechancial systems
are also less efficient during heatwaves, and will consume more energy
than normal. The result is increased demand at a time when the supply of
end-use energy might be lower due to the effect that heat also has on
transmission and distribution equipment. This effect has been documented
in several studies (\#TODO).

Further mechanical cooling is extremely energy intensive. This can be
seen be observing the notable spike in energy use that occurs in nations
across the globe during the cooling months. According to the IEA, air
conditioners and electic fans account for about 20 percent of energy use
builings (\#TODO check this), where buildings account for \#TODO
percentage of global cooling loads @ieaFutureCoolingOpportunities2018.
This number is expected to rapidly incraese to \#TODO as more
populations gain access to mechanical cooling.

Yet the size of the population that does not have access to mechanical
cooling, mainly left out because of income, is quite notable. According
to the International Energy Agency, in warm climates with long and
intense cooling seasons, access to mechanical cooling has a strong
correlation with income, and \#TODO \% of people who would benefit from
mechanical cooling do not have acces to it because of costs. (\#TODO
phrase this better)

The numerous problems associated with mechanical cooling have led
researchers to further investgate opportunities provided by natural
cooling. Following the creation of the air conditioner in the 1950s and
its popularization in the United States in the 1990s, / rise of
modernism and ``the international style'' (\#TODO more research here),
designing for naturally cooled buildings became less prevalent. Building
codes moved toward designs that prioritized the efficiency of heating,
ventilation, and cooling systems at the expense of operable windows and
infiltration. This turn is understandable, given the myriad
uncertainties associated with natural cooling. This includes uncertainty
related to occupant behavior and outdoor climate. Where as mechanically
cooled buildings enable engineers to guarantee a certain temperature up
to single digits of accuracy, naturally cooled buildings rely on the
peculiarities of outdoor temperature and correct operation by occupants
to achieve comfort. This dependence opens the door to extremely
uncomfortable buildings and potential financial losses for building
owners and engineers alike.

Despite these faults, developing mechanisms by which practictioners can
accurately evaluate and potentially integrate natural cooling techniques
into the fast-moving design process is critical (\#TODO explain why!)



\end{document}
